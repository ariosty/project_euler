\documentclass[12pt]{article}

\usepackage{amsmath,amsfonts,amsthm,graphicx,enumerate}

\begin{document}

Let $P_n$ be the set of prime numbers up to $n$, $S_n$ be the set of all prime summation of $n$, a naive (and incorrect!) approach to count $|S_n|$ would be
$$|S_n| = \sum_{p\in P_n} |S_{n-p}|$$
Take 10 as example, the summation $3+7$ would be counted twice, hence the equation is not correct.\\
To avoid repeatedly counting the same summation, one would first divide $S_n$ into several disjoint sets $S_{n,i}$, where $S_{n,i}$ represent the set of prime summation of $n$ with largest part $i$. Summing up the size of all $S_{n,i}$ gives $|S_n|$. Doing so, however, would require $\Theta(n^2)$ space, and thus is suboptimal.\\
Now, let's look at a different perspective. Instead of counting $S_{n,i}$ for fixed $n$ and different $i$, we count $S_{n,i}$ for fixed $i$ and different $n$. In this scenario, only linear space is required, and the program is easier to implement as well.

\end{document}
